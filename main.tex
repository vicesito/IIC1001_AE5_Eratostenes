 
\documentclass{article}
\usepackage{graphicx} % Required for inserting images

\title{Criba de Eratóstenes}
\author{Vicente Cartagena\\Rocío Bocaz}
\date{12-05-2025}

\begin{document} 
\maketitle
La Criba de Eratóstenes es un algoritmo que sirve para encontrar todos los números primos hasta cualquier límite dado.
Nuestro código fue obtenido de la clase de programación IIC1103-8, escrito por la profesora Valeria Herskovic
\subsection{Quién lo inventó?}
Eratóstenes de Cirene, nació en Cirene, 276 a. C. (aunque se estima que pudo haber nacido en 273 a. C.) y falleció Alejandría, 194 a. C.. Fue un polímata, matemático, astrónomo y geógrafo griego. Concibió por primera vez la geografía como una disciplina sistemática, desarrollando una terminología que todavía se utiliza en la actualidad.
\subsection{¿Para qué sirve?}
Sirve para obtener de un modo rápido todos los números primos menores que un número dado. La versión informática de este procedimiento (algoritmo) se ha convertido con los años en un método estándar para caracterizar o comparar la eficacia de diferentes lenguajes de programación.
\end{document}
